\documentclass[11pt, oneside]{article}   	% use "amsart" instead of "article" for AMSLaTeX format
\usepackage{geometry}                		% See geometry.pdf to learn the layout options. There are lots.
\geometry{letterpaper}                   		% ... or a4paper or a5paper or ... 
%\geometry{landscape}                		% Activate for for rotated page geometry
%\usepackage[parfill]{parskip}    		% Activate to begin paragraphs with an empty line rather than an indent
\usepackage{graphicx}				% Use pdf, png, jpg, or eps� with pdflatex; use eps in DVI mode
\usepackage{amsmath}								% TeX will automatically convert eps --> pdf in pdflatex		
\usepackage{amssymb}
\setlength{\parindent}{0cm}

\title{Random Variables}
\author{Yulong Yang}
%\date{}							% Activate to display a given date or no date

\begin{document}
\maketitle
%\section{}
%\subsection{}

\section{Porbability}

\subsection{Conditional probability}

\begin{equation}
P[A|B] = \frac{P[AB]}{P[B]} = \frac{P[B|A]P[A]}{P[B]}
\end{equation}

\subsection{Valid PMF/CDF/PDF}

A valid PDF should have,

\begin{equation}
\forall x, f_{X}(x) \geq 0
\end{equation}

\begin{equation}
F_X(x) = \int_{-\infty}^{x}f_X (u) du
\end{equation}

\begin{equation}
\int_{-\infty}^{\infty}f_X (x) dx = 1
\end{equation}

PMF is simlilar.

A valid CDF should have,

\begin{equation}
    F_X(-\infty) = 0, F_X(\infty) = 1
\end{equation}

\begin{equation}
    \forall x' \geq x, F_X(x') \geq F_X(x)
\end{equation}

\subsection{Inequality}

Given nonnegative RV $X$ (i.e. $P[X<0]=0$) and a constant $c$, \textbf{Markov inequality} shows,

\begin{equation}
    P[X \geq c^2] \leq \frac{E[X]}{c^2}
\end{equation}

Given arbitrary RV $X$ and constant $c>0$, \textbf{Chebyshev inequality} shows,

\begin{equation}
    P[|X - \mu_X| \geq c] \leq \frac{Var[X]}{c^2}
\end{equation}


\subsection{List of common integrals}

\begin{itemize}
    \item $\int \frac{1}{x} dx = ln|x|$
    \item $\int f'(x)e^{f(x)} dx = e^{f(x)}$
    \item $\int 1/(1+x^2)dx = tan^{-1}(x) = arctan(x)$
    \item $\int sin(x) dx = -cos(x), \int cos(x) dx = sin(x)$
    \item $\int a^x dx = a^x/ln(a)$
    \item $\int ln(x) dx = xln(x) - x$
\end{itemize}

\section{Families of RVs}

\subsection{exponential}

Continuous exponential ($\lambda$) RV,

\[ f_X(x) = \left\{ 
  \begin{array}{l l}
    \lambda e^{-\lambda x} & \quad x \geq 0,\\
    0 & \quad \text{otherwise},
  \end{array} \right.\]

with $\lambda > 0$.

\begin{itemize}
    \item $E[X] = 1/ \lambda$
    \item $Var[X] = 1/ \lambda ^2$
\end{itemize}

\subsection{Poisson}

Discrete Poisson ($\alpha = \lambda T$) RV $N(T)$ has PMF,

\[ P_{N(T)}(n) = \left\{ 
  \begin{array}{l l}
      (\lambda T)^n e^{-\lambda T} / n! & \quad n = 0,1,2,...,\\
    0 & \quad \text{otherwise},
  \end{array} \right.\]

  in which $\lambda$ is the arrivals per second, and $T$ is a time interval. If we call the occurrence of the phenomenon of interest an \textit{arrival}, Poisson model could be used to describe the avrage arrival rate of such arrivals. In the time interval specified by $T$, the number of arrivals $N(T)$ has a Poisson PMF with $\alpha=\lambda T$.

A sample use would be, given arrival rate $\lambda$, and time interval $\tau$, calculate the probability that within the time interval the number of arrivals is $n$,

\begin{equation}
P[N(t+\tau) - N(t) = n] = P_{N(\tau)}(n) = \frac{(\lambda \tau)^n e^{-\lambda \tau}}{n!}
\end{equation}

\section{Multiple RVs}

\subsection{joint CDF, PMF and PDF}

\begin{equation}
F_{X,Y}(x,y)=P[X \leq x,Y \leq y]
\end{equation}

\begin{equation}
P_{X,Y}(x,y)=P[X = x,Y = y]
\end{equation}

\begin{equation}
f_{X,Y}(x,y)=\frac{d^2 F_{X,Y}(x,y)} {dx d y}
\end{equation}

\subsection{marginal PMF and PDF}

\begin{equation}
P_{X}(x)=\sum_{y \in S_Y} P_{X,Y}(x,y)
\end{equation}

\begin{equation}
f_{X}(x)=\int_{-\infty}^{\infty}f_{X,Y}(x,y) dy
\end{equation}

\subsection{Expected values}

For 2 RVs $X$ and $Y$, 

\begin{equation}
E[X+Y] = E[X] + E[Y]
\end{equation}

\begin{equation}
Var[X+Y] = Var[X] + Var[Y] + 2E[(X-\mu_X)(Y-\mu_Y)]
\end{equation}

covariance,
\begin{equation}
Cov[X,Y] = E[(X-\mu_X)(Y-\mu_Y)] = r_{X,Y} - \mu_X\mu_Y
\end{equation}

correlation,
\begin{equation}
r_{X,Y} = E[XY]
\end{equation}

correlation coefficient,
\begin{equation}
\rho_{X,Y}= \frac{Cov[X,Y]}{\sqrt{Var[X][Var[Y]}}, -1 \leq \rho_{X,Y} \leq 1
\end{equation}

$X$ and $Y$ are stochastically independent iff,
\begin{equation}
f_{X,Y}(x,y)=f_{X}(x)f_{Y}(y)
\end{equation}


Given $W=g(X,Y)$ and two RVs $X$ and $Y$, the expected value is,

\begin{equation}
Discrete: E[W]=\sum_{x \in S_X}\sum_{y \in S_Y} g(x,y)P_{X,Y}(x,y)
\end{equation}

\begin{equation}
Continuous: E[W]=\int_{-\infty}^{\infty}\int_{-\infty}^{\infty} g(x,y)f_{X,Y}(x,y) dxdy
\end{equation}

\end{document}  
